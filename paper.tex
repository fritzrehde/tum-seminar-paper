\documentclass[sigplan,11pt,nonacm]{acmart}
\settopmatter{printfolios}

\usepackage{booktabs} % For formal tables
\usepackage{subcaption}
\usepackage{tikz}
\usepackage{pgfplots}
\usepackage{pgfplotstable}
\usepackage{hyphenat}
\usepackage{todonotes}
\usepackage[babel]{csquotes}
\usepackage{listings}
\lstset{frame=tb,
  language=Rust,
  aboveskip=3mm,
  belowskip=3mm,
  showstringspaces=false,
  columns=flexible,
  basicstyle={\small\ttfamily},
  numbers=none,
  numberstyle=\tiny\color{gray},
  keywordstyle=\color{blue},
  commentstyle=\color{dkgreen},
  stringstyle=\color{mauve},
  breaklines=true,
  breakatwhitespace=true,
  tabsize=3
}


\begin{document}
\title{Ownership types in theory and practice (in Rust)}
\author{Fritz Rehde}
\affiliation{%
  \institution{Technical University of Munich}
}
\email{fritz.rehde@tum.de}

\begin{abstract}
% Abstract: Brief summary of area, problem, approach, key result

% - why do we need ownership types, what do they do?

Object aliasing is the concept of accessing memory through different symbolic names in object-oriented programming languages.
Many progamming bugs are created through unintentional aliases, which are hard to detect and can lead to unexpected side-effects.
Ownership types are one solution that attempt to prevent many alias-related bugs.
There are different specific implementations of ownership types, but the premise is that 
The premise of ownership types is that not only the fields of an object are protected from external access, but also all (recursively defined) objects stored in those fields.
This is done by allowing objects to take ownership of other objects.
% TODO: add citation\cite{ownership-types-survey}

This paper introduces the different kinds of ownership types and explains how the modern programming language Rust uses ownership types to garantue memory safety.

\end{abstract}

\keywords{Ownership, Type, Safety, Rust}

\maketitle

\section{Introduction}
\label{sec:introduction}
% Introduction: introduce area, problem, approach, key results, contributions, outline


\section{Background}
\label{sec:background}
% Background: if needed, describe prerequisites

\subsection{Memory Safety}

% source: Definition of memory safety\cite{memory1}.
% source: Why memory safery is important\cite{nsa-memory-safety}.


\subsection{State of the art}

% - when/in which language first introduced?
% - why slow adoption?

 % - TODO: source: Standard ML, C++?


\section{Main part}
\label{sec:mainpart}
% Main part (approach, evaluation, discussion, etc.)


\subsection{Ownership types in theory}

% - source: \cite{understanding-evolving-rust}
% - source: \cite{ownership-types-survey}
% Notes from Ownership Types Survey\cite{ownership-types-survey}

% before: simply protect fields of an object from external access
% with OT: also protect objects stored in those fields by claiming ownership of and access to other objects
% problem OTs solve: object aliasing in OOP
% bugs due to unintentional aliases, difficult to find
% potential solutions: banning aliases, clearly advertising aliases, manage and control its effects
% page 16:
% 1998: flexible alias protection:
% limit visibility of changes to objects via aliases
% by limiting where alias could propagate
% by limiting the changes that could be observed through alias
% didn't feature "this"
% evolved into original OT:
% owners-as-dominators invariant:
% - heap divided into next regions (ownership contexts)
% - OC is set of objects
% - organised hierarchically: each owned object inside its owner, nesting relationship
% - each object has one single context
% - receives permissions to reference objects in other contexts through type

% core concept: object ownership: objects owner by other object or entities (global owners, stack-based owners)
% objects in same ownership context: siblings or peers

% control aliasing by removing aliasing
% unique references: only reference to object in system
% borrowing: 
% - reference to owned object temporarily passed to another object (for method call)
% - method over => all temporary references vanish/unusable

% control aliasing by controlling effects of aliasing
% immutability: state cannot change

% owners-as-dominators
% - program heap is tree structured, object is inside its owner
% - external access only through interface, i.e. owner
% - rules for a to validly access b: 1. a is owner of b; 2. a and b are siblings; 3. b is outside a
% - "a is inside b": "a owns b": a = b or a is owned by b, transitive
% - "a is outside b": converse of inside: "b is inside a"
% - advantages: simple, clear and strong garantee, easy to reason about properties of code
% - disadvantages: no aliasing => no common idioms; programming more difficult

% owners-as-modifiers
% - weaker form of owners-as-dominators, allows read-only references
% - read-only reference: only used to read fields, call pure methods (may not modify any existing object, including receiver)
% - aliasing unrestricted, but only owner can modify
% - usecase: requirements from verification of object-oriented programs
% - rules for a to validly reference b through reference r: either
% 1. a is owner of b
% 2. a and b are siblings
% 3. b is outside a
% 4. r is read-only reference and only pure methods can be called

% ownership domains:

% owners-as-ombudsmen:
% - problem: strong topological requirement: tree structure: some idioms don't fit in tree structure
% - allow multiple objects to define a shared aggregate owner
% - rules for a to validly reference b through reference r: either
% 1. a is owner of b
% 2. a and b are siblings
% 3. b is outside a
% 4. a is owned by aggregate owner b



% QUESTIONS:
% - difference between aliases and references
% - difference between "a is owner of b" and "b is outside a"

\subsection{Implementation in Rust}
% - borrow checker

% Notes from rust-book\cite{rust-book}.
% Ownership definition: set of rules that govern how Rust program manages memory
% Compiler checks rules, if violated program doesn't compile
% Purpose of Rust ownership is to mainly manage heap data
% Ownership rules:
% - each value in Rust has an owner
% - there can only be one owner at a time
% - when the owner goes out of scope, the value will be dropped (value is valid in scope, invalid out of scope)
% Definition scope:

% \begin{lstlisting}
% {                      // s is not valid here, it’s not yet declared
%     let s = "hello";   // s is valid from this point forward
%     // do stuff with s
% }                      // this scope is now over, and s is no longer valid
% \end{lstlisting}

% data types of known size at compile time (int, float, bool, char): stored on stack, popped off stack when out of scope, quickly and trivially copied

% - TODO: more sources


\subsection{Practical examples using Rust}
% - show common Rust ownership errors
% - Protection against memory vulnerabilitie examples
% - reference: \cite{rust-by-example}


\subsection{Alternatives}
% - comparison to alternative type systems without ownership (C/C++: none at all, Java: garbage collection)
% - choose non-ownership programming language e.g. C++20
% - compare code examples
% - compile time differences between Rust and non-ownership lang
% - runtime differences between Rust and non-ownership lang


\section{Related work}
\label{sec:relatedwork}
% (In a paper: Related Work – might come before main part)


\section{Summary \& Outlook}
\label{sec:summary}



% QUESTIONS:
% Online article: Should I cite the actual pdf or the website that brought me to the pdf

\bibliographystyle{ACM-Reference-Format}
\bibliography{paper} % read paper.bib file

\end{document}
