\documentclass[sigplan,11pt,nonacm]{acmart}
\settopmatter{printfolios}

\usepackage{booktabs} % For formal tables
\usepackage{subcaption}
\usepackage{tikz}
\usepackage{pgfplots}
\usepackage{pgfplotstable}
\usepackage{hyphenat}
\usepackage{todonotes}
\usepackage[babel]{csquotes}
\usepackage{listings}
\lstset{frame=tb,
  language=Rust,
  aboveskip=3mm,
  belowskip=3mm,
  showstringspaces=false,
  columns=flexible,
  basicstyle={\small\ttfamily},
  numbers=none,
  numberstyle=\tiny\color{gray},
  keywordstyle=\color{blue},
  commentstyle=\color{dkgreen},
  stringstyle=\color{mauve},
  breaklines=true,
  breakatwhitespace=true,
  tabsize=3
}


\begin{document}
\title{Ownership types in theory and practice (in Rust)}
\author{Fritz Rehde}
\affiliation{%
  \institution{Technical University of Munich}
}
\email{fritz.rehde@tum.de}

\begin{abstract}
% Abstract: Brief summary of area, problem, approach, key result

% - why do we need ownership types, what do they do?

Object aliasing is the concept of accessing the same memory through different symbolic names in object-oriented programming languages.
Many progamming bugs are created through unintentional aliases, which are hard to detect and can lead to unexpected side-effects.
Ownership types are one solution that attempt to prevent many alias-related bugs.
The premise of ownership types is that not only the fields of an object are protected from external access, but also all objects stored in those fields.
This is done by allowing objects to take ownership of other objects.
% TODO: add citation\cite{ownership-types-survey}

This paper depicts the different kinds of ownership types and explains how the modern programming language Rust uses ownership types to garantue memory safety.

\end{abstract}

\keywords{Ownership, Type, Safety, Rust}

\maketitle

\section{Introduction}
\label{sec:introduction}
% Introduction: introduce area, problem, approach, key results, contributions, outline

% TODO: write something

\section{Background}
\label{sec:background}
% Background: if needed, describe prerequisites

One of the main goals of ownership types is to improve memory safety.
Understanding memory-related bugs will, therefore, help in understanding why ownership types are useful.

\subsection{Memory Safety}

% source: Definition of memory safety\cite{memory1}.
% source: Why memory safery is important\cite{nsa-memory-safety}.

Memory safety is defined as ...

\subsection{Domain specific terms}

In the following, some terms that commonly used throughout this paper are explained.
The core concept of ownership types is that objects can be \emph{owners} of other objects.
Given objects \emph{a} and \emph{b}, \emph{a} is \emph{inside} \emph{b} if \emph{a} is the same object as \emph{b} or \emph{a} is owned by \emph{b}, transitively.
The \emph{outside} relation is the converse of the \emph{inside} relation.

In most ownership types variants, the heap is divided into different regions called ownership contexts.
Each object has one single ownership context, which is the set of all objects it owns.
Objects that are in the same ownership context are called \emph{siblings} or \emph{peers}.
\cite{ownership-types-survey}

\subsection{State of the art}

% - when/in which language first introduced?
% - why slow adoption?


Potential solutions to the aliasing problem include banning aliases altogether, clearly advertising aliases or managing and controlling its effects\cite{ownership-types-survey}.
Many different flavours of ownership types have been explored.
In this paper, state of the art implementations of the general concept such as owners-as-dominators, owners-as-modifiers and owners-as-ombudsmen will be explored further.
These approaches differ in how much, if any, references/aliases they allow.


\section{Main part}
\label{sec:mainpart}
% Main part (approach, evaluation, discussion, etc.)

\subsection{Ownership types in theory}


\subsection{Owners-as-dominators}

One such flavour is the owners-as-dominators\cite{ownership-types-survey} invariant, where the heap is structured as a tree with each object being inside its owner.
External access is therefore only possible through the owner.



\subsection{Implementation in Rust}


\subsection{Practical examples using Rust}
% - show common Rust ownership errors
% - Protection against memory vulnerabilitie examples
% - reference: \cite{rust-by-example}


\subsection{Alternatives}
% - comparison to alternative type systems without ownership (C/C++: none at all, Java: garbage collection)
% - choose non-ownership programming language e.g. C++20
% - compare code examples
% - compile time differences between Rust and non-ownership lang
% - runtime differences between Rust and non-ownership lang


\section{Related work}
\label{sec:relatedwork}
% (In a paper: Related Work – might come before main part)


\section{Summary \& Outlook}
\label{sec:summary}



\bibliographystyle{ACM-Reference-Format}
\bibliography{paper} % read paper.bib file

\end{document}
